\documentclass[10pt,onecolumn]{book}

\usepackage{times} % font
\usepackage{graphicx} % picture reference
\usepackage{amsmath} % split of mathematical formulas
\usepackage{amssymb} % special mathematical symbols
\usepackage{geometry} % margin
\usepackage{setspace} % letter-spacing
\usepackage{indentfirst} % indent
\geometry{left=2cm,right=2cm, top=2cm, bottom=2cm}
\usepackage{hyperref} % hyperlink
\usepackage{cite}
\usepackage[sectionbib]{chapterbib}

\usepackage{multirow}
\usepackage{color}
\usepackage{ulem}
\usepackage{todonotes}

%\usepackage{fancy}%页眉页脚包
%\pagestyle{plain}%页眉页脚设置
\usepackage{fancyhdr}
\pagestyle{fancy}


\def\ie{\emph{i.e.}}
\def\eg{\emph{e.g.}}
\def\etal{\em {et al.}}

\newcommand{\bm}[1]{\mbox{\boldmath{$#1$}}}
\newcommand{\figref}[1]{Fig. \ref{#1}}
\newcommand{\tabref}[1]{Tab. \ref{#1}}
\newcommand{\equref}[1]{(\ref{#1})}
\newcommand{\secref}[1]{Sect. \ref{#1}}
\newcommand{\algref}[1]{Alg. \ref{#1}}
\newcommand{\myPara}[1]{\vspace{.05in}\noindent\textbf{#1}}
\newcommand{\rev}[1]{\textcolor{blue}{#1}}
\newcommand{\rr}[1]{\textcolor{red}{#1}}
\newcommand{\cg}[1]{\textcolor{green}{#1}}
\newcommand{\bb}[1]{\textcolor{blue}{#1}}
\newcommand{\bl}[1]{\textbf{#1}}
\newcommand{\ul}[1]{\underline{#1}}
\newcommand{\mc}[1]{\mathcal{#1}}
\newcommand{\mb}[1]{\mathbb{#1}}

\begin{document}
\date{}

\title{\textbf{Paper Reading}}
\author{Jinming Su}

\maketitle

\thispagestyle{empty}
\newpage
\pagenumbering{Roman}
\newpage
\tableofcontents
%\newpage
%\listoffigures
%\newpage
%\listoftables
%\newpage
%\pagenumbering{arabic}
\newpage
\listoftodos

\newpage
\pagenumbering{arabic}
\mainmatter

\chapter{Salient Object Detection}
\section{DNA: Deeply-supervised Nonlinear Aggregation for Salient Object Detection, arxiv, 2019.}
\begin{figure}[h]
\centering
\includegraphics[width=0.4\textwidth]{figures_paper_reading/DNA_Deeply-supervised_Nonlinear_Aggregation_for_Salient_Object_Detection.png}
\caption{Residual Learning.}
\label{fig:1-1_residual_learning}
\end{figure}

This paper has two contributions: 
(1) \uline{theoretically and experimentally} analyze the natural limitaion of traditional side-output aggregration which can only make limited use of multi-scale side-ouput informantion; 
(2) propose Deeply-supervised nonlinear aggregration (DNA) for side-output features. 
(3) As experience, in DNA, convolution layers with kernels of $n \times 1$ and $1 \times n$ are used, which is proved to be effective. Moreover, authers claim that large kernel size in DNA can improve performance.


{\small
\bibliographystyle{plain}
\bibliography{Ref_paper_reading}
}

\end{document}
